\documentclass[a4paper,10pt]{jsarticle}
\usepackage{amsmath,amssymb,mathrsfs,amsthm}
\usepackage{framed,color}
\usepackage{newtxtext,newtxmath}
% ---Set margin---
\setlength{\textheight}{\paperheight}
\setlength{\topmargin}{4.6truemm}
\addtolength{\topmargin}{-\headheight}
\addtolength{\topmargin}{-\headsep}
\addtolength{\textheight}{-60truemm}

% Theorem Environments ---------------------------------------------------
\theoremstyle{definition}
\newtheorem{theorem}{Theorem}
\newtheorem{definition}{Definition}
\newtheorem{lemma}{Lemma}
\newtheorem{proposition}{Proposition}
\newtheorem{corollary}{Corollary}


% ShortCut Environments ---------------------------------------------------
\newcommand{\eq}[1]{\begin{align}#1\end{align}}
\newcommand{\items}[1]{\begin{itemize}#1\end{itemize}}
\newcommand{\enums}[1]{\begin{enumerate}#1\end{enumerate}}
\newcommand{\pmat}[1]{\begin{pmatrix}#1\end{pmatrix}}
\newcommand{\point}[1]{\subsubsection*{#1}\addcontentsline{toc}{subsubsection}{#1}}
\newcommand{\dcup}[1][]{\operatorname*{\cup_\mathnormal{#1}}}
\newcommand{\dcap}[1][]{\operatorname*{\cap_\mathnormal{#1}}}
%
\begin{document}
\section{Radon-Nikodymの定理}
Capinski, and Kopp, \textit{Measure, Integral and Probability}, Springer, 2004 をもとにRadon-Nikodymの定理を証明:
可測空間$(\Omega,\mathcal{F})$上の$F\in\mathcal{F}$に対する測度$\nu(F)$が
\eq{F\mapsto\nu(F)=\int_Ffd\mu}
となるような$f$を見つける問題.
\subsection*{用語}
\subsubsection*{絶対連続(absolutely continuous)}
任意の$F\in\mathcal{F}$に対して$\mu(F)=0$ならば$\nu(F)=0$,が成り立つならば,
$\nu$は$\mu$に対して\textbf{絶対連続}であるといい,$\nu\ll\mu$と表す.
\subsubsection*{押さえる(dominate)}
任意の$F\in\mathcal{F}$に対して$0\le\nu(F)\le\mu(F)$が成り立つとき,$\mu$は$\nu$を\textbf{押さえる}という.
\subsubsection*{分割(partition)}
$\mathcal{F}$内の有限な排反部分集合の集まり$\mathcal{P}+\left(F_i\right)_{i\le n}$で$\cup_iF_i=\Omega$をみたすものを,(有限可測な)$\Omega$の\textbf{分割}と呼ぶ.
\subsubsection*{細分(refinement)}
$2$つの分割$\mathcal{P}$,$\mathcal{P}'$について,任意の$\mathcal{P}$の要素が$\mathcal{P}'$の排反な要素の和集合で表されるとき,$\mathcal{P}'$は$\mathcal{P}$の\textbf{細分}と呼ぶ.
\subsubsection*{$\sigma$-有限($\sigma$-finite)}
$\cup_iF_i=\Omega$をみたす$\mathcal{F}$-可測な集合列$F_i$が存在して,各$i$について$\nu(F_i)$が有限の値をとるとき,$\nu$を$\sigma$\textbf{-有限}な測度と呼ぶ.
\subsection*{証明}
Radon-Nikodymの定理を証明する前に,以下の補助定理を証明しておくと便利.
Radon-Nikodymの定理との違いは
\items{
	\item $2$つの測度が$\sigma$-有限ではなく,一方が片方を押さえているという仮定になっている.
	\item 押さえている方の測度$\mu$が全測度で$1$となる.}
という$2$点.ただし,結論の形式はRadon-Nikodymの定理と同様なので,その雰囲気は伝わるはず.
\begin{theorem}
任意の$F\in\mathcal{F}$に対して,$\mu(\Omega)=1$,$0\le\nu(F)\le\mu(F)$が成り立つ,つまり$\mu$は$\nu$を押さえる測度とする.
このとき,任意の$F\in\mathcal{F}$に対して,
\eq{\nu(F)=\int_Fhd\mu\label{rep1}}
をみたす,($\Omega$上の)非負$\mathcal{F}$-可測関数$h$が存在する.
\end{theorem}
以下のステップで証明
\items{
	\item 分割$\mathcal{P}$に含まれる集合上で,(\ref{rep1})をみたすような$h_{\mathcal{P}}$を構成する.
		ついでに$\mathcal{P}_{n+1}$が$\mathcal{P}_n$を細分するような分割の列$\mathcal{P}_1,\mathcal{P}_2,...$について,
		\eq{\int_\Omega h^2_{\mathcal{P}_n}d\mu\label{nondecint}}
		が非減少列になることを確認する.
	\item 1.の結果,および(\ref{nondecint})で定められる列は上限$1$で押さえられることから,収束定理を用いて(\ref{rep1})をみたす$h$を$h_{\mathcal{P}_n}$の極限として求めることができる.
	\item 2.の方法で定めた$h$が望ましい性質を持つことを確認する.}
\end{document}