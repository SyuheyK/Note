\documentclass[a4paper,10pt]{jsarticle}
\usepackage{amsmath,amssymb,mathrsfs,amsthm}
\usepackage{framed,color}
\usepackage{newtxtext,newtxmath}
 \usepackage{ascmac}
% ---Set margin---
\setlength{\textheight}{\paperheight}
\setlength{\topmargin}{4.6truemm}
\addtolength{\topmargin}{-\headheight}
\addtolength{\topmargin}{-\headsep}
\addtolength{\textheight}{-60truemm}

% Theorem Environments ---------------------------------------------------
\theoremstyle{definition}
\newtheorem{theorem}{Theorem}
\newtheorem{definition}{Definition}
\newtheorem{lemma}{Lemma}
\newtheorem{proposition}{Proposition}
\newtheorem{corollary}{Corollary}


% ShortCut Environments ---------------------------------------------------
\newcommand{\eq}[1]{\begin{align}#1\end{align}}
\newcommand{\items}[1]{\begin{itemize}#1\end{itemize}}
\newcommand{\enums}[1]{\begin{enumerate}#1\end{enumerate}}
\newcommand{\pmat}[1]{\begin{pmatrix}#1\end{pmatrix}}
\newcommand{\point}[1]{\subsubsection*{#1}\addcontentsline{toc}{subsubsection}{#1}}
\newcommand{\dcup}[1][]{\operatorname*{\cup_\mathnormal{#1}}}
\newcommand{\dcap}[1][]{\operatorname*{\cap_\mathnormal{#1}}}
%
\begin{document}
\setcounter{section}{1}
%%%%%%%%%
%Chapter.2
%%%%%%%%%
\section{順伝播型ネットワーク}
ニューラルネットワークを定式化
%
%section2.1
\subsection{ユニットの出力}
\begin{itembox}[l]{キーワード}
順伝播型(ニューラル)ネットワーク,多層パーセプトロン,重み,バイアス,活性化関数
\end{itembox}
ネットワークを構成する各ユニットは,入力$x_1,...,x_I$を受けとり,総入力:
\eq{u=w_1x_1+\cdots+w_Ix_I+b}
を計算した上で,\text{活性化関数}$f$を用いて
\eq{z=f(u)}
を出力します.
ここで$w_1,...,w_I$は\textbf{重み},$b$は\textbf{バイアス}です.

$1$ つの層に複数のユニットが存在している場合,ベクトルを用いて次のように書けます.
\eq{\mathbf{u}&=\mathbf{Wx}+\mathbf{b}\\
\mathbf{z}&=\mathbf{f}(\mathbf{u})}
ただし,入力(=第一層)のユニット数は$I$,第二層のユニット数は$J$とし,各記号は次のとおり:
\eq{\mathbf{u}=(u_1,...,u_J)^\top,\mathbf{x}=(x_1,...,x_I)^\top,\mathbf{b}=(b_1,...,b_J)^\top,\mathbf{z}=(z_1,...,z_J)^\top,\\
\mathbf{W}=\begin{pmatrix}(w_{11}&\cdots&w_{1I}\\\vdots&\ddots&\vdots\\w_{J1}&\cdots&w_{JI}\\\end{pmatrix},\mathbf{f}(\mathbf{u})=(f(u_1),...,f(u_J))^\top}

%
%section2.2
\subsection{活性化関数(ユニット間,出力層ではない)}
\begin{itembox}[l]{キーワード}
ロジスティック(シグモイド)関数,シグモイド関数,正規化線形関数,マックスアウト
\end{itembox}
\items{
	\item ロジスティック関数 $\subset$ シグモイド関数 
		\eq{f(u)=\frac{1}{1+e^{-u}}}
	\item 双曲線正接関数(hyperbolic tangent) $\subset$ シグモイド関数 
		\eq{f(u)=\tanh(u)}
	\item 正規化線形関数
		\eq{f(u)=\max(u,0)}
		この関数をもつユニットを\textbf{ReLU} と言います.
	\item 線形写像・恒等写像
	\item ロジスティック関数を区分線形近似したもの
		\eq{f(u)=\begin{cases}-1&u<-1\\u&-1\le u<1\\1&u\ge1\end{cases}}
	\item マックスアウト($K$個のユニットをまとめたようなもの)
		\eq{
			u_{jk}&=\sum_iw_{jik}z_i+b_{jk}\\
			f(u_j)&=\max_{k+1,...,K}u_{jk}}
	}
%
%section2.3
\subsection{多層ネットワーク}
\begin{itembox}[l]{キーワード}
入力層,中間層(=隠れ層),出力層
\end{itembox}
各層のユニットの入出力の計算を区別するために,層が$l=1,2,...,L$となっているときに各変数の肩に$(l)$と記すことにします.
すると,層$2$のユニットの出力$\mathbf{z}^{(2)}$は,入力層$\mathbf{x}$から
\eq{
	\mathbf{u}^{(2)}&=\mathbf{W}^{(2)}\mathbf{x}+\mathbf{b}^{(2)}\\
	\mathbf{z}^{(2)}&=\mathbf{f}(\mathbf{u}^{(2)})}
と計算され,その後の層$l+1$のユニットの出力$\mathbf{z}^{(l+1)}$は,$1$つ下の層の出力$\mathbf{z}^{(l)}$から
\eq{
	\mathbf{u}^{(l+1)}&=\mathbf{W}^{(l+1)}\mathbf{z}^{(l)}+\mathbf{b}^{(l+1)}\\
	\mathbf{z}^{(l+1)}&=\mathbf{f}(\mathbf{u}^{(l+1)})}
と計算される.最後に層$L$まで計算が終わったら,ネットワークの最終的な出力を
\eq{\mathbf{y}=\mathbf{z}^{(L)}}
と表記する.
各層の重み$\mathbf{W}^{(l)}$とバイアス$\mathbf{b}^{(l)}$をユニットのパラメータと呼び,
出力$\mathbf{y}$は入力とパラメータによって決まることから,
\eq{\mathbf{y}\left(\mathbf{x};\mathbf{W}^{(2)},...,\mathbf{W}^{(L)},\mathbf{b}^{(2)},...,\mathbf{b}^{(L)}\right)}
あるいは簡単に
\eq{\mathbf{y}(\mathbf{x};\mathbf{w})}
と書きます.
%
%section2.4
\subsection{出力層の設計と誤算関数}
\begin{itembox}[l]{キーワード}
訓練サンプル,訓練データ,誤算関数,回帰,二値分類,多クラス分類,最尤推定,ソフトマックス関数,交差エントロピー
\end{itembox}
学習に使用する入出力データのペア
\eq{\left\{(\mathbf{x}_1,\mathbf{d}_1),(\mathbf{x}_2,\mathbf{d}_2),...,(\mathbf{x}_N,\mathbf{d}_N)\right\}}
を訓練データといいます.
%
\subsubsection{回帰}
出力層の活性化関数には恒等写像を用い,
誤算関数には二乗誤算の和
\eq{E(\mathbf{w}&=\frac{1}{2}\sum_{n=1}^N||\mathbf{d}_n-\mathbf{y}(\mathbf{x}_n;\mathbf{w})||^2}
を使用し,$E(\mathbf{w})$がもっとも小さくなる$\mathbf{w}$を求める.
%
\subsubsection{二値分類}
ここでは$\mathbf{x}$を指定したときに$d=1$となる事後確率$p(d=1|\mathbf{x})$を,
ネットワーク全体の出力$y(\mathbf{x};\mathbf{w})$でモデル化し,
出力層の活性化関数にロジスティック関数を使うことで,
\eq{p(d=1|\mathbf{x})\approx y(\mathbf{x};\mathbf{w})}
とします.
パラメータ$\mathbf{w}$は最尤推定で求めることにします.
$p(d=1|\mathbf{x})=y(\mathbf{x};\mathbf{w})$より$p(d=0|\mathbf{x})=1-y(\mathbf{x};\mathbf{w})$であることと,
\eq{p(d|\mathbf{x})=p(d=1|\mathbf{x})^dp(d=0|\mathbf{x})^{1-d}}
より,$N$個の入力に対する尤度は
\eq{L(\mathbf{w})=\prod_{n=1}^Np(d_n|\mathbf{x}_n;\mathbf{w})=\prod_{n=1}^Ny(\mathbf{x}_n;\mathbf{w})^{d_n}\left\{1-y(\mathbf{x}_n;\mathbf{w})\right\}^{1-d_n}}
となります.誤差関数としてはこれに対数をとって符号を反転させた
\eq{E(\mathbf{w})=-\sum_{n=1}^N\left[d_n\log y(\mathbf{x}_n;\mathbf{w})+(1-d_n)\log\left\{1-y(\mathbf{x}_n;\mathbf{w})\right\}\right]}
を用て,パラメータ$\mathbf{w}$を求めます.
%
\subsubsection{多クラス分類}
ネトワークの出力層にクラス数$K$と同じ$K$個のユニットを並べ,出力層の各ユニット$k(=1,...,K)$の出力を
\eq{y_k=z_k^{(L)}=\frac{\exp(u_k^{(L)})}{\sum_{j=1}^K\exp(u_j^{(L)})}}
とします.

\end{document}

















